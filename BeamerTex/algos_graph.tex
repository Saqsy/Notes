%%%%%%%%%%%%%%%%%%%%%%%%%%%%%%%%%%%%%%%%%%%%%%%%%%%%%%%%%%%%%%
\begin{frame}[fragile]
\frametitle{Sidebar : $P=NP$ ?}
\begin{columns}[T]

\begin{column}{0.45\textwidth}
\begin{itemize}[<+->]
\item The P versus NP problem is a major unsolved problem in theoretical computer science.
\item It asks whether every problem whose solution can be quickly verified can also be quickly solved.
\item e.g.the Subset Sum problem — for a set of numbers and a target sum, determine if there exists a subset of numbers that adds up to the target sum.
\item Easy to verify a solution: For $\{3,4,5,6,7\}$ and a target of $9$, a valid subset is $\{4,5\}$. This is $P$.
\item But computing it takes $O(2^n)$ time (non-polynomial) $NP$.
\end{itemize}
\end{column}

\pause
\begin{column}{0.45\textwidth}
\begin{itemize}[<+->]
\item But maybe $NP$ algorithms are `hard' simply because we haven't found a better solution?
\item It is thought that $P \neq NP$, meaning there are problems that can't be solved in polynomial time, but for which the answer could be verified in polynomial time.
\item A proof either way would have profound implications for mathematics, cryptography, AI etc.
\end{itemize}
\end{column}

\end{columns}
\end{frame}

%%%%%%%%%%%%%%%%%%%%%%%%%%%%%%%%%%%%%%%%%%%%%%%%%%%%%%%%%%%%%%
\begin{frame}[fragile]
\frametitle{Algorithms : TSP on Graphs}
\begin{columns}[T]

\begin{column}{0.45\textwidth}
\begin{itemize}[<+->]
\item Imagine planning a delivery route around a graph, starting from a particular vertex.
\item What's the least cost by which you can visit every vertex without ever returning to one~?
\item Finding the optimal path (to reduce travelling time) is an $NP$ problem.
\item For small graphs you could do this exhaustively, but for very large graphs this combinatorial approach becomes untenable.
\item One `greedy' approach is to simply go to your closest unvisited neighbour each time.
\end{itemize}
\end{column}

\pause
\begin{column}{0.45\textwidth}
\begin{itemize}[<+->]
\item Typically gives results within \verb^25%^
of the optimal solution, but sometimes give a worst-case solution $\ldots$
\item 
{\scriptsize \verb^A -> B -> C -> D -> J -> I -> E -> F -> G^}

\begin{center}
\includegraphics[width=0.5\textwidth]{../Images/Linkedg.pdf}
\end{center}
\end{itemize}
\end{column}

\end{columns}
\end{frame}

%%%%%%%%%%%%%%%%%%%%%%%%%%%%%%%%%%%%%%%%%%%%%%%%%%%%%%%%%%%%%%

\begin{frame}[fragile]
\frametitle{TSP II}
\begin{columns}[T]

\begin{column}{0.45\textwidth}
\lstinputlisting[style=basicc,linerange={5-26},numbers=none]{../../ADTs/Graph/Indep/indep.c}
\end{column}

\pause
\begin{column}{0.45\textwidth}
\lstinputlisting[style=basicc,linerange={27-46},numbers=none]{../../ADTs/Graph/Indep/indep.c}
\end{column}

\end{columns}
\end{frame}



%%%%%%%%%%%%%%%%%%%%%%%%%%%%%%%%%%%%%%%%%%%%%%%%%%%%%%%%%%%%%%

\begin{frame}[fragile]
\frametitle{Algorithms : Dijkstra on Graphs}
\begin{columns}[T]

\begin{column}{0.45\textwidth}
\begin{itemize}[<+->]
\item It's often important to find the shortest path through a graph from one vertex to another.
\item One way of doing this is the greedy algorithm due to Dijkstra discovered $~1956$.
\item Picks the unvisited vertex with the lowest distance, \& calculate the distance through it to each unvisited neighbor, updating the neighbour's distance if smaller.
\item Mark visited when done with neighbors.
\end{itemize}
\end{column}

\begin{column}{0.45\textwidth}
\begin{center}
\pause
\includegraphics[width=0.4\textwidth]{../Images/dijkstra1.pdf}
\pause
\includegraphics[width=0.4\textwidth]{../Images/dijkstra2.pdf}
\pause
\includegraphics[width=0.4\textwidth]{../Images/dijkstra3.pdf}
\end{center}
\end{column}

\end{columns}
\end{frame}

%%%%%%%%%%%%%%%%%%%%%%%%%%%%%%%%%%%%%%%%%%%%%%%%%%%%%%%%%%%%%%
